\documentclass[dvipsnames]{article}
\usepackage{hyperref}
\usepackage[utf8]{inputenc}
\usepackage[usenames,dvipsnames,svgnames,table]{xcolor}
\usepackage{color}
\usepackage{alltt}
\usepackage{amsmath}
\usepackage{amssymb}
\usepackage{listings}
\usepackage{xcolor}
\usepackage{multicol}
\usepackage{graphicx}
\usepackage{tikz}

\setlength{\textwidth}{6.5in}
\setlength{\textheight}{8.9in}
\setlength{\voffset}{-1in}
\setlength{\oddsidemargin}{0in}
\setlength{\evensidemargin}{0in}

\begin{document}

\noindent Criptografia (2018/2019)
\hfill DCC/FCUP

\begin{center}\LARGE\bf 
  Implementação de Gerador de Chaves RSA em Python \\
\end{center}

\vskip 0.4cm
\hrule
\vskip 0.4cm

\section{Introdução}
Pretende-se neste relatório descrever a Implementação de uma função em python, \texttt{genRSAkey(l)} que receba como argumento um tamanho de chave $\ell$ e retorne um tuplo da forma $(n,p,q,e,d)$, respeitando as condições descritas a seguir:
\begin{enumerate}
  \item $n=pq$ sendo $p$ e $q$ primos e por forma a que $log_2(n) \geq \ell$ (ou seja que $n$ tenha na sua representação binária pelo menos $\ell$ bits);
  \item $e$ inteiro coprimo com $(p-1)(q-1)$, ou seja, o máximo divisor comum entre $e$ e $(p-1)(q-1)$ seja 1;
  \item $d$, por forma a que $ed \equiv 1 (mod (p-1)(q-1))$
\end{enumerate}

\section{Geração de candidatos a primos $p$ e $q$}
Usando o módulo \texttt{random} do Python, a função \texttt{genRand(l)} gera números ímpares no intervalo $2^{w-1} + 1$ e $2^{w} - 1$ em que $w=\lfloor \frac{\ell}{2} \rfloor$, sendo que todos os números passíveis de serem gerados por \texttt{random.randrange} no intervalo descrito acima são depois \textbf{OR}-ed com $1$ no seu bit menos significativo, aumentando assim a eficiência do gerador já que isto elimina a necessidade de gerar números candidatos a primos e depois testar num ciclo \texttt{while} a divisibilidade destes por $2$, visto que qualquer primo $>2$ é necessariamente ímpar.

\vskip 0.4cm

\subsection{Crivo de Eratóstenes}

Depois de gerado o número $p$ (candidato a primo), verifcar-se-à, numa primeira fase, contra um crivo de Eratóstenes, gerado apenas uma vez aquando a chamada de \texttt{genRSAkey(l)} e guardado em \texttt{small\_primes}, contendo os $1000$ primeiros números primos. Se algum destes é factor de $p$, então sabemos que $p$ não é primo, sendo assim necessário invocar \texttt{genRand(l)} novamente e repetir o teste.

\subsection{Teste de Miller-Rabin}

No caso em que o número $p$ gerado passa pelo processo de \textit{sieving} supra descrito sem serem encontrados factores primos, prosseguimos para o teste de primalidade de Miller-Rabin (probabilístico). É de salientar que este último domina a complexidade temporal de \texttt{genRSAkey(l)}. Usar \textit{sieving} é eficaz em minimizar o número de vezes em que é gerado um número composto $p$ e este tem que ser sujeito ao teste de Miller-Rabin, ou seja: se p tem factores primos relativamente pequenos, detectamos desde logo que este é composto, gerando assim um novo $p$ cuja primalidade volta a ser testada pelo mesmo processo, minimizando assim o número de iterações "desnecessárias" do teste de M-R. Formalmente, dada uma constante $B$, só serão sujeitos ao teste de M-R os canidatos a primos que se não se revelem $B$\textit{-smooth} \textbf{TODO: INCLUIR FONTE SOBRE B-SMOOTH} e neste caso $B$ é o último elemento do crivo de Eratóstenes.

\vskip 0.4cm

\noindent O valor de \texttt{max\_rounds} em \texttt{genRSAkey(l)}, corresponde ao número máximo de testemunhas aleatórias $a$ geradas no intervalo $[2,n-2]$ com relação à primalidade de um inteiro $n$ ímpar tal que $n>3$. Já que o teste de M-R retorna \texttt{False} quando $n$ é composto, porque foi gerado um $a$ que é testemunha da existência de factores primos de $n$, então podemos dizer que se $n$ é um número composto, quanto mais iterações do teste de M-R forem feitas para esse $n$, maior é a probabilidade de encontrar uma testemunha a (gerada aleatóriamente) que respeite esses critérios e determine que $n$ não é primo.

\vskip 0.4cm

\noindent Baseando-nos no parágrafo anterior, podemos concluir que: se $n$ é um número ímpar composto e ao fim de \texttt{max\_rounds} o teste M-R não retornou \texttt{False}, então $n$ é um "falso primo".\\

\noindent Sejam as variáveis aleatórias:

\begin{itemize}
  \item $Y_k \leftarrow n$ é declarado primo depois de $k$ iterações do teste de M-R;
  \item $X \leftarrow n$ é um número composto (sendo $\overline{X} \leftarrow n$ é um número primo).
\end{itemize}

% Consultar: https://en.wikipedia.org/wiki/Miller–Rabin_primality_test

\noindent Por \cite{DBLP:journals/dcc/Lenstra00} e assumindo a veracidade da Hipótese Generalizada de Riemann é possivel mostrar que pelo menos $\frac{3}{4}$ das testemunhas aleatórias em $a \in [2,n-2]$ podem garantir que $n$ é composto, que é o mesmo que dizer que no máximo $\frac{1}{4}$ das testemunhas aleatórias no mesmo intervalo não garantem tal. Com isto podemos dizer que $Pr[Y_k|X] \leq \frac{1}{4}$ para uma iteração do teste de M-R, ou seja $k=1$. Sendo que para todas as $k$ iterações do teste de M-R, é independente a escolha de $a$, mostrou-se em \cite{1204.1657v2} que a probabilidade de erro deste teste pode ser descrita em função de $k$ como $Pr[X|Y_k] \leq (\frac{1}{4})^k$ e que $Pr[Y_k|X]$ é relacionável com $Pr[X|Y_k]$, advindo do teorema de Bayes e conforme detalhado em \cite{1709.09963}.

\vskip 0.4cm

\noindent Em vista dos resultados obtidos escolheu-se um $k=100$, coincidente com \texttt{max\_rounds} na função de geração de tuplos para as chaves RSA, o que admite uma probabilidade de erro igual ou inferior a $\frac{1}{4^{100}}$, algo admissível na geração dos factores primos de $n$ (RSA \textit{modulus}). Para suportar o argumento acima, de acordo com \cite{FIPS} (página 70) é dito que para uma probabilidade de erro até $2^{-112}$, são recomendadas pelo menos $56$ iterações do teste M-R, quando $p$ e $q$ tiverem na sua representação 2048 bits.

\subsection{Noção de \textit{strong primes} em RSA}

Apesar do módulo \texttt{random} na linguagem Python poder ser usado como função pseudo-aleatória para gerar grandes números candidatos a primos, é descrito na documentação \cite{pyrandom} que este não é adequado para usos em aplicações criptográficas. Em vista dos possíveis riscos que surgem de nos afastarmos da verdadeira aleatoriedade

\section{Escolha do expoente público $e$}

\section{Cálculo de $d$}

\section{Tempos de execução e conclusões finais}

\bibliography{report}
\bibliographystyle{ieeetr}

\end{document}
